\documentclass[12pt,a4paper]{article}
\usepackage[utf8]{inputenc}
\usepackage[english]{babel}
\usepackage{amsmath}
\usepackage{amsfonts}
\usepackage{amssymb}
\usepackage{graphicx}
\usepackage[left=2cm,right=2cm,top=2cm,bottom=2cm]{geometry}
\author{Ethan Kelly}
\title{Introducing features of agency into computational models of infectious disease}
\date{27 May 2021, 13:00}
\begin{document}
\maketitle

\begin{centering}
{\it Abstract for talk given at the First University of Glasgow Computational Biology Conference.}\\
\end{centering}

\vspace{5pt}

Computational models of disease often begin with games on graphs (also known as networks). For instance, some work has been done using the Firefighter Problem (referred to as {\scshape Firefighter}) as a basis for epidemic modelling. This work has its limitations in producing realistic models, which we propose can be solved by introducing features of agency.

In {\scshape Firefighter}, vertices in a graph are thought of as trees and somewhere on the graph a tree catches fire. We can `defend' a vertex in our role as a firefighter, preventing it from catching fire. The fire then spreads to all adjacent vertices not currently defended. We can then deploy another defence, burning occurs and so on until either the fire can't spread further or we can't defend anything else. By swapping `fire' for `infection' and `tree' for `individual', we quickly have a rudimentary model for disease spread. This, however is where current work becomes sparse. What we aim to discuss is how to extend these models with a focus on agency-related features. In particular, we will discuss how using ideas from compartmental modelling approaches in graph models can result in a ready-made agency framework for use in {\scshape Firefighter} and whether we could use a dynamical systems approach for such an extended model.

\hspace{2in}

{\bf Key points of discussion:}
\begin{enumerate}
\item Current computational approaches to graph models of disease
\item Extending existing graph models to better account for individual agency
\item Introducing a compartmental modelling approach
\item Algorithmically assessing the feasibility of a dynamical systems approach to compartmental graph models
\end{enumerate}

\end{document}